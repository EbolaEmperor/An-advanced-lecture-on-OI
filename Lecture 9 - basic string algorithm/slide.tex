%%%%%%%%%%%%%%%%%%%%%%%%%%%%%%%%%%%%%%%%%%%%%%%%%%%%%%
% A Beamer template for Ritsumeikan University       %
% Author: Ming-Hao Xu (Xu Minghao)                   %
% Date:   April 2022.                                %
% LPPL Licensed.                                     %
%%%%%%%%%%%%%%%%%%%%%%%%%%%%%%%%%%%%%%%%%%%%%%%%%%%%%%

\documentclass{beamer}
\usepackage{hyperref}

\usepackage[UTF8]{ctex}
\usepackage[T1]{fontenc}

% other packages
\usepackage{latexsym,amsmath,xcolor,multicol,booktabs,calligra}
\usepackage{graphicx,pstricks,listings,stackengine}
\usefonttheme[onlymath]{serif}

% dummy text; remove it when working on this template
\usepackage{lipsum}

\author{Ebola}
\title{字符串进阶:扩展kmp}
\institute{
    Institute of Mathematics, \\
    Zhejiang University.
}
\date{Jan, 2024}
\usepackage{Ritsumeikan}

% defs
\def\cmd#1{\texttt{\color{red}\footnotesize $\backslash$#1}}
\def\env#1{\texttt{\color{blue}\footnotesize #1}}
\definecolor{deepblue}{rgb}{0,0,0.5}
\definecolor{deepred}{rgb}{0.6,0,0}
\definecolor{deepgreen}{rgb}{0,0.5,0}
\definecolor{halfgray}{gray}{0.55}

\lstset{
    basicstyle=\ttfamily\tiny,
    keywordstyle=\bfseries\color{deepblue},
    emphstyle=\ttfamily\color{deepred},    % Custom highlighting style
    stringstyle=\color{deepgreen},
    numbers=left,
    numberstyle=\small\color{halfgray},
    rulesepcolor=\color{red!20!green!20!blue!20},
    frame=shadowbox,
}


\begin{document}

\begin{frame}
    \titlepage
\end{frame}

\begin{frame}
    \tableofcontents[sectionstyle=show,subsectionstyle=show/shaded/hide,subsubsectionstyle=show/shaded/hide]
\end{frame}

\section{基础回顾}

\begin{frame}{kmp算法}
    \small

    kmp算法解决的是单文本串、单模式串匹配问题,即:
    给定文本串 $S$ 和模式串 $T$,求 $T$ 在 $S$ 中完整出现的所有位置。

    \vspace{1em}
    复杂度是 $O(n)$.
\end{frame}

\begin{frame}[fragile]{kmp算法}
    \footnotesize
    我们来回顾一下 kmp 算法的流程。
    首先有一个 next 数组,我这里把它写成 $\pi$,定义如下:
    \begin{equation*}
        \pi(i)=\max \{k\;|\; T[1,...,k]=T[i-k+1,...,i],\; k=0,...,i-1\}
    \end{equation*}

    \vspace{1em}\pause
    我们把 $T$(长$m$) 和 $S$(长$n$,$n>m$) 放到一起:
    \verb|A = T#S|
    
    \vspace{1em}
    对于一个位置 $i$,如果 $\pi(i)=m$,说明 $T$ 从 $i-m+1$ 位置开始出现;
    反过来,如果 $T$ 从 $i-m+1$ 位置开始出现,那么一定 $\pi(i)=m$。
    
    \vspace{1em}
    我们只要找到 $\pi(i)=m$ 的位置,然后就知道答案了。
\end{frame}

\begin{frame}[fragile]{kmp算法}
    \footnotesize
    现在考虑如何求 $\pi$ 数组。

    最暴力的求法,$O(n^3)$:

    \begin{lstlisting}[language=c++]
for(int i = 1; i <= n; i++){
    int k;
    for(k = i-1; k >= 0; k--)
        if(子串(1,...,k) == 子串(i-k+1,...,i))
            break;
    pi[i] = k;
}
    \end{lstlisting}
\end{frame}

\begin{frame}[fragile]{kmp算法}
    \footnotesize
    我们观察到:$\pi(i+1)\leq \pi(i)+1$,为什么?(举例说明)

    \vspace{1em}\pause
    借助这个观察,我们可以优化代码:
    \begin{lstlisting}[language=c++]
for(int i = 1; i <= n; i++){
    int k;
    for(k = pi[i-1]+1; k >= 0; k--)
        if(子串(1,...,k) == 子串(i-k+1,...,i))
            break;
    pi[i] = k;
}
    \end{lstlisting}

    这个复杂度是 $O(n^2)$
\end{frame}

\begin{frame}[fragile]{kmp算法}
    \footnotesize
    现在我们看这个集合:
    \begin{equation*}
        \mathcal{P}(i)=\{k\;|\; T[1,...,k]=T[i-k+1,...,i],\; k=0,...,i-1\}.
    \end{equation*}
    注意到 $\pi(i)=\max \mathcal{P}(i)$.

    \vspace{1em}\pause
    又注意到 $\pi(i)-1\in\mathcal{P}(i-1)$(为什么?)。
    \pause 那么 $k$ 循环可以只遍历 $\mathcal{P}(i-1)$ 里的数,像这样:

    \begin{lstlisting}[language=c++]
pi[1] = 0;
for(int i = 2; i <= n; i++){
    int k;
    for(k = max(P(i-1)); k >= 0; k=P(i-1)里面比k小的那个数)
        if(A[k+1] == A[i])
            break;
    pi[i] = k + 1;
}
    \end{lstlisting}

    \vspace{1em}\pause
    注意到 $\mathcal{P}(i)$ 里面第二大的数是 $\pi(\pi(i))$,
    第三大的数是 $\pi(\pi(\pi(i)))$,…… (为什么?)
    所以上面的 $k$ 循环很好实现。总复杂度 $O(n)$.
\end{frame}

\begin{frame}{[HNOI2008] GT考试}

    \small 
    阿申准备报名参加 GT 考试,准考证号为 $N$ 位数$X_1,X_2…X_n\ (0\le X_i\le 9)$,他不希望准考证号上出现不吉利的数字。
    
    \vspace{1em}
    他的不吉利数字$A_1,A_2,\cdots, A_m\ (0\le A_i\le 9)$ 有 $M$ 位,不出现是指 $X_1,X_2\cdots X_n$ 中没有恰好一段等于 $A_1,A_2,\cdots ,A_m$,$A_1$ 和 $X_1$ 可以为 $0$。

    \vspace{1em}
    阿申想知道不出现不吉利数字的号码有多少种,输出模 $K$ 取余的结果。

    \vspace{1em}
    $N\leq10^9$,$M\leq 20$,$K\leq1000$。
\end{frame}

\begin{frame}{[HNOI2008] GT考试}
    \small 
    设 $f_{i,j}$ 表示当前考虑到第 $i$ 位,其中末尾和 $A_1...A_m$ 匹配了 $j$ 位。

    设 $g_{j,k}$ 表示当前末尾匹配了 $j$ 位,如果添加一个数字后能够匹配 $k$ 位,
    有多少种添加数字的方案。这是一个可以预处理的数组。

    \vspace{1em}\pause
    我们得到转移方程:
    \begin{equation*}
        f_{i,j}=\sum_{k=0}^{m-1} f_{i-1,k}g_{k,j}.
    \end{equation*}

    显然可以用矩阵快速幂优化(应该都会吧?)。

    \vspace{1em}\pause
    $g$ 数组可以用 kmp 来算。枚举当前匹配长度 $j$,再枚举下一个数字 $c$,
    用 next 数组算一下添加 $c$ 之后末尾匹配长度是多少(记为 $k$),然后令 $g_{j,k}++$.
\end{frame}

\section{扩展 kmp (Z 函数)}

\begin{frame}{扩展 kmp (Z 函数)}
    \small

    对于一个长度为 $n$ 的字符串,定义 $z_i$ 表示 $s$ 与 $s[i...n]$ 的最长公共前缀长度。
    这就是 \textbf{Z 函数}。

    \vspace{1em}
    在研究如何求 Z 函数之前,我们先来看看 Z 函数的应用。
\end{frame}

\begin{frame}[fragile]{字符串匹配}
    \small

    给定文本串 $S$ 和模式串 $T$,求 $T$ 在 $S$ 中完整出现的所有位置。

    \pause\vspace{1em}
    【解】令 \verb|A=T#S|,求出 $A$ 的 Z 函数,然后找到所有 $z_i=|T|$ 的位置即可。
\end{frame}

\begin{frame}[fragile]{本质不同的子串}
    \small

    给定文本串 $S$,现在往 $S$ 的开头添加一个字母 $c$,
    问增加了几个本质不同的子串。

    \pause\vspace{1em}
    【解】求 $cS$ 的 Z 函数,取最大值 $z_\text{max}$,显然,
    长度超过 $z_\text{max}$ 的前缀都是新增的本质不同子串(反之,
    长度不超过 $z_\text{max}$ 的前缀都不是新增的本质不同子串)。
\end{frame}

\begin{frame}[fragile]{字符串的最小周期}
    \small
    给定一个长度为 $n$ 的字符串 $S$,
    找到其最短的整周期,
    即寻找一个最短的字符串 $T$,
    使得 $S$ 可以被若干个 $T$ 拼接而成的字符串表示。

    \pause\vspace{1em}
    【解】求 $S$ 的 Z 函数,找到最小的 $n$ 的因数 $i$,满足 $i+z_i=n$.
\end{frame}

\begin{frame}
    \begin{center}
        {\Huge\calligra Thank You}
    \end{center}
\end{frame}

\end{document}