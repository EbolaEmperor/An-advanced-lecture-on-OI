%%%%%%%%%%%%%%%%%%%%%%%%%%%%%%%%%%%%%%%%%%%%%%%%%%%%%%
% A Beamer template for Ritsumeikan University       %
% Author: Ming-Hao Xu (Xu Minghao)                   %
% Date:   April 2022.                                %
% LPPL Licensed.                                     %
%%%%%%%%%%%%%%%%%%%%%%%%%%%%%%%%%%%%%%%%%%%%%%%%%%%%%%

\documentclass{beamer}
\usepackage{hyperref}

\usepackage[UTF8]{ctex}
\usepackage[T1]{fontenc}

% other packages
\usepackage{latexsym,amsmath,xcolor,multicol,booktabs,calligra}
\usepackage{graphicx,pstricks,listings,stackengine}
\usefonttheme[onlymath]{serif}

% dummy text; remove it when working on this template
\usepackage{lipsum}

\author{Ebola}
\title{字符串进阶:AC自动机}
\institute{
    Institute of Mathematics, \\
    Zhejiang University.
}
\date{Jan, 2024}
\usepackage{Ritsumeikan}

% defs
\def\cmd#1{\texttt{\color{red}\footnotesize $\backslash$#1}}
\def\env#1{\texttt{\color{blue}\footnotesize #1}}
\definecolor{deepblue}{rgb}{0,0,0.5}
\definecolor{deepred}{rgb}{0.6,0,0}
\definecolor{deepgreen}{rgb}{0,0.5,0}
\definecolor{halfgray}{gray}{0.55}

\lstset{
    basicstyle=\ttfamily\tiny,
    keywordstyle=\bfseries\color{deepblue},
    emphstyle=\ttfamily\color{deepred},    % Custom highlighting style
    stringstyle=\color{deepgreen},
    numbers=left,
    numberstyle=\small\color{halfgray},
    rulesepcolor=\color{red!20!green!20!blue!20},
    frame=shadowbox,
}


\begin{document}

\begin{frame}
    \titlepage
\end{frame}

\begin{frame}
    \tableofcontents[sectionstyle=show,subsectionstyle=show/shaded/hide,subsubsectionstyle=show/shaded/hide]
\end{frame}

\section{基础回顾}

\begin{frame}{字典树(Trie)}
    \small

    字典树把所有的字符串存储在一棵树中,
    可以方便地查询所有的前缀。

    \vspace{1em}
    我们来回顾一下字典树模板题。
\end{frame}

\begin{frame}[fragile]{字典树(Trie)}
    \small

    插入操作
    \begin{lstlisting}[language=c++]
int mapping(char c){
    if(c>='A' && c<='Z') return c-'A';
    else return c-'a'+26;
}
void insert(char s[]){
    int n = strlen(s+1);
    int cur = 0;
    sz[0]++;
    for(int i = 1; i <= n; i++){
        int j = mapping(s[i]);
        if(ch[cur][j]==0){
            ch[cur][j] = tot;
            tot++;
        }
        cur = ch[cur][j];
        sz[cur]++;
    }
}
    \end{lstlisting}
\end{frame}

\begin{frame}[fragile]{字典树(Trie)}
    \small

    查询操作
    \begin{lstlisting}[language=c++]
int query(char s[]){
    int cur = 0;
    int n = strlen(s+1);
    for(int i = 1; i <= n; i++){
        int j = mapping(s[i]);
        if(ch[cur][j]==0) return 0;
        cur = ch[cur][j];
    }
    return sz[cur];
}
    \end{lstlisting}
\end{frame}

\begin{frame}{最大异或和问题}
    \small

    最大异或和问题是 Trie 的一个经典应用。

    \vspace{1em}
    给定 $n\;(\leq 10^5)$ 个数,所有数均不超过 $2^{31}-1$.
    从中选两个数,使它们异或起来最大。 
\end{frame}

\begin{frame}{最大异或和问题}
    \small

    把所有的数都转化为 $31$ 位二进制数,高位不足则补零,
    然后把二进制数当成字符串插入进 Trie 中。

    \vspace{1em}\pause
    现在,我们枚举 $x=a_i\;(i=1,...,n)$,来找一个数 $a_j$,使它和 $x$ 异或起来最大。

    \vspace{1em}\pause
    从高位到低位贪心,尽可能让异或和的高位为 $1$。例如:
    如果 $x$ 最高位是 $0$,那么我们希望选出的 $a_j$ 最高位是 $1$,
    这样异或起来最高位才会是 $1$,因此我们第一步从 Trie 的根节点
    往 $1$ 的方向走。
\end{frame}

\begin{frame}[fragile]{最大异或和问题}
    \small

    总之,如果 $x$ 的第 $k$ 位是 $x_{k}$,那么这一步就尽量往 $x_{k}\;\text{xor}\; 1$
    方向走,除非 Trie 不存在对应的分支,此时不得不往 $x_{k}$ 走。
    最后代码像这样:
    \begin{lstlisting}[language=c++]
int query(int x){
    int cur = 0;
    const int n = 31;
    for(int i = 1; i <= n; i++){
        int j = (x >> (31-i)) & 1;
        if(ch[cur][j^1]==0) cur = ch[cur][j];
        else cur = ch[cur][j^1];
    }
    return val[cur];
}
    \end{lstlisting}
\end{frame}

\begin{frame}[fragile]{第 k 大异或和问题}
    \small

    如果想对于给定的 $x$,找到一个 $a_i$,使 $x\oplus a_i$ 是
    $x\oplus a_1,...,x\oplus a_n$ 中第 $k$ 大的数,应该如何写?

    \pause
    \begin{lstlisting}[language=c++]
int query(int x, int k)
{
    int o=1;
    int res=0;
    for(int i=31;i>=0;i--)
    {
        int j=(x>>i)&1;
        if(sz[ch[o][j^1]]>=k) o=ch[o][j^1],res|=1u<<i;
        else k-=sz[ch[o][j^1]],o=ch[o][j];
    }
    return res;
}
    \end{lstlisting}
\end{frame}

\begin{frame}[fragile]{[十二省联考 2019] 异或粽子}
    \small

    给定 $n\;(\leq 5\times 10^5)$ 个数 $a_1,...,a_n$,选一个区间 $[l,r]$,将 $a_l,...,a_r$ 全部异或起来,
    得到这个区间的权值。求权值前 $m\;(\leq 2\times 10^5)$ 大的区间权值之和。
\end{frame}

\begin{frame}[fragile]{[十二省联考 2019] 异或粽子}
    \small

    令 $b_i=a_1\oplus...\oplus a_i$,那么 $a_l\oplus...\oplus a_r=b_{l-1}\oplus b_r$,转化为两个数的异或,
    可以用 Trie 解决。

    \vspace{1em}\pause
    具体地,先把 $b_0,...,b_n$ 的二进制插入 Trie 中。
    接下来对每个 $b_i$ 找到一个 $b_j$ 使它和 $b_i$ 异或起来最大。

    \vspace{1em}\pause
    将这些值存进一个优先队列中。每次从优先队列取出最大的异或和,弹出,
    如果它是 $b_i$ 和其它数的异或和中第 $k$ 大的,
    就找到 $b_i$ 和其它数的异或和中第 $k+1$ 大的加入优先队列。
    不断重复,直到弹出的数达到 $2m$ 个为止,最后答案要除以 $2$. (为什么?)
\end{frame}

\begin{frame}
    \begin{center}
        {\Huge\calligra Thank You}
    \end{center}
\end{frame}

\end{document}