%%%%%%%%%%%%%%%%%%%%%%%%%%%%%%%%%%%%%%%%%%%%%%%%%%%%%%
% A Beamer template for Ritsumeikan University       %
% Author: Ming-Hao Xu (Xu Minghao)                   %
% Date:   April 2022.                                %
% LPPL Licensed.                                     %
%%%%%%%%%%%%%%%%%%%%%%%%%%%%%%%%%%%%%%%%%%%%%%%%%%%%%%

\documentclass{beamer}
\usepackage{hyperref}

\usepackage[UTF8]{ctex}
\usepackage[T1]{fontenc}

% other packages
\usepackage{latexsym,amsmath,xcolor,multicol,booktabs,calligra}
\usepackage{graphicx,pstricks,listings,stackengine}
\usefonttheme[onlymath]{serif}

% dummy text; remove it when working on this template
\usepackage{lipsum}

\author{Ebola}
\title{高级计数技巧:生成函数与多项式运算}
\institute{
    Institute of Mathematics, \\
    Zhejiang University.
}
\date{July, 2023}
\usepackage{Ritsumeikan}

% defs
\def\cmd#1{\texttt{\color{red}\footnotesize $\backslash$#1}}
\def\env#1{\texttt{\color{blue}\footnotesize #1}}
\definecolor{deepblue}{rgb}{0,0,0.5}
\definecolor{deepred}{rgb}{0.6,0,0}
\definecolor{deepgreen}{rgb}{0,0.5,0}
\definecolor{halfgray}{gray}{0.55}

\lstset{
    basicstyle=\ttfamily\tiny,
    keywordstyle=\bfseries\color{deepblue},
    emphstyle=\ttfamily\color{deepred},    % Custom highlighting style
    stringstyle=\color{deepgreen},
    numbers=left,
    numberstyle=\small\color{halfgray},
    rulesepcolor=\color{red!20!green!20!blue!20},
    frame=shadowbox,
}


\begin{document}

\begin{frame}
    \titlepage
\end{frame}

\begin{frame}
    \tableofcontents[sectionstyle=show,subsectionstyle=show/shaded/hide,subsubsectionstyle=show/shaded/hide]
\end{frame}

\section{生成函数}

\subsection{普通生成函数}

\begin{frame}{定义与例子}

对于一个数列$a_0,a_1,a_2,...$,它的普通生成函数为:
\begin{equation*}
    F(x)=\sum_n a_nx^n
\end{equation*}

\pause
\begin{itemize}
    \item $\{1,2,3\}$的普通生成函数为:$1+2x+3x^2$
    \item $\{1,1,1,...\}$的普通生成函数为:$\sum_{n=0}^\infty x^n$
\end{itemize}

\end{frame}

\begin{frame}{封闭形式}
    考虑$\{1,1,1,...\}$的普通生成函数:
    \begin{equation*}
        F(x)=\sum_{n=0}^\infty x^n
    \end{equation*}

    \pause
    注意到
    \begin{equation*}
        xF(x)+1=F(x)
    \end{equation*}

    因此
    \begin{equation*}
        F(x)=\frac{1}{1-x}
    \end{equation*}

    我们称这种没有求和符号的表达式为封闭形式
\end{frame}

\begin{frame}{封闭形式}
    求数列$\{1,0,1,0,...\}$的普通生成函数,并化为封闭形式

    \vspace{1em}\pause
    【解】
    \begin{equation*}
        F(x)=\sum_{n=0}^\infty x^{2n}=\frac{1}{1-x^2}
    \end{equation*}
\end{frame}

\begin{frame}{封闭形式}
    求数列$\{1,2,3,4,...\}$的普通生成函数,并化为封闭形式

    \vspace{1em}\pause
    【解】
    \begin{align*}
        F(x)&=\sum_{n=0}^\infty (n+1)x^{n}\\
        &= \sum_{n=0}^\infty (x^{n+1})'\\
        &= \left(\sum_{n=0}^\infty x^{n}\right)'\\
        &= \left(\frac{1}{1-x}\right)'\\
        &= \frac{1}{(1-x)^2}
    \end{align*}
\end{frame}

\begin{frame}{组合计数例子}
    \small
    假设你去买水果,一共要买$n$个,其中苹果只能买偶数个,西瓜最多买一个,梨可以买任意个。请问一共有多少种购买方案?

    \vspace{1em}\pause
    【解】设买$i$个苹果的方案数为$a_i$,事实上$a_i$仅在$i$为偶数时为$1$,否则为$0$,因此苹果的生成函数为:
    \begin{equation*}
        F(x)=1+x^2+x^4+...=\frac{1}{1-x^2}
    \end{equation*}

    \pause
    同理,西瓜和梨的生成函数为
    \begin{equation*}
        G(x)=1+x,\quad H(x)=1+x+x^2+...=\frac{1}{1-x}
    \end{equation*}

    \pause
    相乘得到
    \begin{equation*}
        F(x)G(x)H(x)=\frac{1}{(1-x)^2}=\sum_{n=0}^\infty (n+1)x^{n}
    \end{equation*}

    所以一共有$n+1$种购买方案。
\end{frame}

\subsection{指数生成函数}

\begin{frame}{定义与例子}

对于一个数列$a_0,a_1,a_2,...$,它的指数生成函数为:
\begin{equation*}
    F(x)=\sum_n a_n\frac{x^n}{n!}
\end{equation*}

\pause
\begin{itemize}
    \item $\{1,2,3\}$的指数生成函数为:$1+x+\frac{1}{2}x^2$
    \item $\{1,1,1,...\}$的指数生成函数为:$\sum_{n=0}^\infty \frac{x^n}{n!}$
\end{itemize}

\end{frame}

\begin{frame}{封闭形式}
    考虑$\{1,1,1,...\}$的指数生成函数:
    \begin{equation*}
        F(x)=\sum_{n=0}^\infty \frac{x^n}{n!}=e^x
    \end{equation*}
\end{frame}

\begin{frame}{封闭形式}
    求数列$\{1,0,1,0,...\}$的指数生成函数,并化为封闭形式

    \vspace{1em}\pause
    【解】
    \begin{align*}
        F(x)&=\sum_{n=0}^\infty \frac{x^{2n}}{(2n)!}\\
        &= \sum_{n=0}^\infty \frac{1+(-1)^n}{2} \frac{x^{n}}{n!}\\
        &= \frac{1}{2} \sum_{n=0}^\infty \frac{x^n}{n!} - \frac{1}{2} \sum_{n=0}^\infty \frac{(-x)^n}{n!}\\
        &= \frac{1}{2}\left(e^x+e^{-x}\right)
    \end{align*}
\end{frame}

\begin{frame}{例题选讲:森林计数}
    求$n$个点带标号、深度不超过$k$的森林一共有多少种。
\end{frame}

\section{快速数论变换(NTT)}

\begin{frame}{问题引入}
    给定多项式
    \begin{align*}
        f(x) &= a_0 + a_1x + a_2x^2 + \cdots + a_nx^n\\
        g(x) &= b_0 + b_1x + b_2x^2 + \cdots + b_mx^m
    \end{align*}
    求$h(x)=f(x)g(x)$的各项系数,对$998244353$取模。
\end{frame}

\begin{frame}{原根}
    注意:$998244353=119 \times 2^{23} + 1$

    \vspace{1em}
    我们说$3$是$998244353$的原根,因为$3^1,3^2,...,3^{998244352}$ 对$998244353$取模的结果两两不同。下面为了方便我们记$p=998244353$
    
    \vspace{1em}\pause
    根据费马小定理,$3^{p-1}\equiv 1\;(\text{mod}\;p)$,那么令$\omega_n=3^{119\times 2^{24-l}}$(其中$n=2^l$),我们会发现$\omega_{n}^{n}=1$,所以$\omega_n$可以作为$n$次单位根!

    \vspace{1em}\pause
    把FFT中的运算全部换成取模意义下的运算,再把$n$次单位复根替换成这里的$\omega_n$,就得到了NTT,它的性质就是取模意义下的FFT。
\end{frame}

\begin{frame}[fragile]{代码}
    \begin{lstlisting}{language=c++}
void NTT(int *a,bool INTT)
{
    for(int i=0;i<len;i++) r[i]=(r[i/2]/2)|((i&1)<<(l-1));
    for(int i=0;i<len;i++) if(i<r[i]) swap(a[i],a[r[i]]);
    for(int i=1;i<len;i<<=1)
    {
        int p=(i<<1);
        int wn=Pow(3,(Mod-1)/p);
        if(INTT) wn=Pow(wn,Mod-2);
        for(int j=0;j<len;j+=p)
        {
            int w=1;
            for(int k=0;k<i;k++)
            {
                int x=a[j+k],y=1ll*w*a[i+j+k]%Mod;
                a[j+k]=(x+y)%Mod;
                a[i+j+k]=(x-y+Mod)%Mod;
                w=1ll*w*wn%Mod;
            }
        }
    }
    //为了方便,我们通常把INTT的最后一步除以n也写进NTT函数里
    if(INTT) for(int i=0;i<len;i++) a[i]=1ll*a[i]*inv%Mod;
}
    \end{lstlisting}
\end{frame}

\section{多项式运算}

\begin{frame}{多项式牛顿迭代}

\small
给定多项式$g(x)$,求一个多项式$f(x)$,满足
\begin{equation}
    g(f(x))\equiv 0 \quad (\text{mod}\;x^n)
\end{equation}

注意:$(\text{mod}\;x^n)$的意思是只保留最低的$n$项。

\vspace{1em}\pause

考虑倍增。设$f_0(x)$是方程(1)在$(\text{mod}\;x^{\left\lceil\frac{n}{2}\right\rceil})$意义下的解,那么$f(x)-f_0(x)$的最低次项就是$x^{\left\lceil\frac{n}{2}\right\rceil}$项。考虑泰勒展开:
\begin{align*}
    g(f(x))&\equiv \sum_{i=0}^\infty \frac{g^{(i)}(f_0(x))}{i!}(f(x)-f_0(x))^i\quad (\text{mod}\;x^n)\\
    &\equiv g(f_0(x))+g'(f_0(x))(f(x)-f_0(x))\quad (\text{mod}\;x^n)
\end{align*}

\pause
由方程(1)得:
\begin{equation*}
    f(x)\equiv f_0(x) -\frac{g(f_0(x))}{g'(f_0(x))} \quad (\text{mod}\;x^n)
\end{equation*}

\end{frame}

\begin{frame}{多项式求逆}
    \small
    给定一个多项式$f(x)$,求一个多项式$g(x)$,使得
    \begin{equation}
        f(x)g(x) \equiv 1 \quad (\text{mod} x^n)
    \end{equation}

    \vspace{1em}\pause
    用多项式牛顿迭代的思路。设
    \begin{equation*}
        h(g(x))=\frac{1}{g(x)}-f(x)
    \end{equation*}

    \pause
    设$g_0(x)$是方程(2)在$(\text{mod}\;x^{\left\lceil\frac{n}{2}\right\rceil})$意义下的解,由牛顿迭代有:
    \begin{align*}
        g(x)&\equiv g_0(x) -\frac{h(g_0(x))}{h'(g_0(x))} \quad (\text{mod}\;x^n)\\
        &\equiv g_0(x) +\frac{\frac{1}{g_0(x)}-f(x)}{\frac{1}{g_0^2(x)}} \quad (\text{mod}\;x^n)\\
        &\equiv 2g_0(x)-g_0^2(x)f(x)\quad (\text{mod}\;x^n)
    \end{align*}

    FFT优化即可,复杂度$T(n)=T(n/2)+O(n\log n)=O(n\log n)$
\end{frame}

\begin{frame}{多项式开方}
    \small
    给定一个多项式$f(x)$满足$[x^0]f(x)=1$,求一个多项式$g(x)$,使得
    \begin{equation}
        g^2(x) \equiv f(x) \quad (\text{mod} x^n)
    \end{equation}
    
    \vspace{1em}\pause
    用多项式牛顿迭代的思路。设
    \begin{equation*}
        h(g(x))=g^2(x)-f(x)
    \end{equation*}

    \pause
    设$g_0(x)$是方程(3)在$(\text{mod}\;x^{\left\lceil\frac{n}{2}\right\rceil})$意义下的解,由牛顿迭代有:
    \begin{align*}
        g(x)&\equiv g_0(x) -\frac{h(g_0(x))}{h'(g_0(x))} \quad (\text{mod}\;x^n)\\
        &\equiv g_0(x) -\frac{g_0^2(x)-f(x)}{2g_0(x)} \quad (\text{mod}\;x^n)\\
        &\equiv \frac{1}{2}g_0(x)+\frac{1}{2}g_0^{-1}(x)f(x)\quad (\text{mod}\;x^n)
    \end{align*}

    需要做一次多项式求逆和一次多项式乘法,复杂度$T(n)=T(n/2)+O(n\log n)=O(n\log n)$
\end{frame}

\begin{frame}{多项式除法(取模)}
    \small
    给定一个$n$次多项式$f(x)$和一个$m(\leq n)$次多项式$g(x)$,求多项式$Q(x),R(x)$,使得
    \begin{equation}
        f(x) \equiv Q(x)g(x)+R(x) \quad (\text{mod} x^n)
    \end{equation}

    且$\text{deg}\;R<m$(类似整数的带余除法)

    \vspace{1em}\pause
    考虑构造变换:
    \begin{equation*}
        f^R(x)=x^{n}f\left(\frac{1}{x}\right)
    \end{equation*}

    \pause
    将方程(4)中的$x$替换成$x^{-1}$,并且在两边乘上$x^n$,得到:
    \begin{equation*}
        x^nf\left(\frac{1}{x}\right)=x^{n-m}Q\left(\frac{1}{x}\right)x^m g\left(\frac{1}{x}\right) + x^{n-m+1}x^{m-1}R\left(\frac{1}{x}\right)
    \end{equation*}

    \pause
    \begin{equation*}
        \implies f^R(x) = Q^R(x)g^R(x)+x^{n-m+1}R^R(x)
    \end{equation*}
\end{frame}

\begin{frame}{多项式除法(取模)}
    \small
    \begin{equation*}
        f^R(x) = Q^R(x)g^R(x)+x^{n-m+1}R^R(x)
    \end{equation*}

    如果两边模掉$x^{n-m-1}$,就可以消除$R^R(x)$项的影响,而$Q^R(x)$的次数为$(n-m)<(n-m+1)$,所以$Q^R(x)$不受影响\pause,可见
    \begin{equation*}
        f^R(x)\equiv Q^R(x)g^R(x)\quad (\text{mod}\; x^{n-m+1})
    \end{equation*}

    多项式求逆得到$Q^R(x)$,反转得到$Q(x)$,回代方程(4)求出$R(x)$
\end{frame}

\begin{frame}{多项式ln}
    \small
    给定一个多项式$f(x)$,满足$[x^0]f(x)=1$,求一个多项式$g(x)$,使得
    \begin{equation}
        g(x) \equiv \ln f(x) \quad (\text{mod} x^n)
    \end{equation}
    
    \vspace{1em}\pause
    求导得
    \begin{equation*}
        g'(x) \equiv f'(x)f^{-1}(x) \quad (\text{mod} x^n)
    \end{equation*}

    依次进行求导、多项式求逆、多项式乘法、积分即可,复杂度$O(n\log n)$
\end{frame}

\begin{frame}{多项式exp}
    \small
    给定一个多项式$f(x)$,保证$[x^0]f(x)=0$,求一个多项式$g(x)$,使得
    \begin{equation}
        g(x) \equiv e^{f(x)}=\sum_{k=0}^\infty \frac{f^{k}(x)}{k!} \quad (\text{mod} x^n)
    \end{equation}
    
    \vspace{1em}\pause
    用多项式牛顿迭代的思路。设
    \begin{equation*}
        h(g(x))=\ln g(x)-f(x)
    \end{equation*}

    \pause
    设$g_0(x)$是方程(3)在$(\text{mod}\;x^{\left\lceil\frac{n}{2}\right\rceil})$意义下的解,由牛顿迭代有:
    \begin{align*}
        g(x)&\equiv g_0(x) -\frac{h(g_0(x))}{h'(g_0(x))} \quad (\text{mod}\;x^n)\\
        &\equiv g_0(x) -\frac{\ln g_0(x)-f(x)}{g_0^{-1}(x)} \quad (\text{mod}\;x^n)\\
        &\equiv g_0(x) \left(1-\ln g_0(x) + f(x)\right)\quad (\text{mod}\;x^n)
    \end{align*}

    需要做一次多项式ln和一次多项式乘法,复杂度$T(n)=T(n/2)+O(n\log n)=O(n\log n)$
\end{frame}

\begin{frame}{多项式的幂}
    \small
    给定正整数$k$和一个多项式$f(x)$,保证$[x^0]f(x)=1$,求一个多项式$g(x)$,使得
    \begin{equation}
        g(x) \equiv \left(f(x)\right)^k \quad (\text{mod} x^n)
    \end{equation}
    
    \vspace{1em}\pause
    注意到
    \begin{equation*}
        g(x)=e^{k\ln f(x)}
    \end{equation*}
\end{frame}

\begin{frame}{多项式三角函数}
    \small
    给定一个多项式$f(x)$,求$(\text{mod}\; x^n)$意义下的$\sin f(x),\cos f(x),\tan f(x)$.
    
    \vspace{1em}\pause
    根据欧拉公式$e^{ix}=\cos x + i \sin x$,得到
    \begin{align*}
        \sin x &= \frac{e^{ix}-e^{-ix}}{2i}\\
        \cos x &= \frac{e^{ix}+e^{-ix}}{2}
    \end{align*}

    \pause 将$f(x)$代入得到
    \begin{align*}
        \sin f(x) &= \frac{\exp{(if(x))}-\exp{(-if(x))}}{2i}\\
        \cos f(x) &= \frac{\exp{(if(x))}+\exp{(-if(x))}}{2}\\
        \tan f(x) &= \sin f(x) \left(\cos f(x)\right)^{-1}
    \end{align*}

    \pause
    注意,在对$998244353$取模的意义下,我们取$i^2\equiv -1\;(\text{mod}\; 998244353)$,可以取$i=86583718$或$911660635$
\end{frame}

\begin{frame}{例题选讲:小朋友和二叉树}
    \small
    给定一个$n$个数的集合$c=\{c_1,c_2,...,c_n\}$,其中$1\leq c_i\leq M \leq 10^5$,如果一棵二叉树的每个点权都在集合$c$中,我们就叫它“好树”。现给定$m\leq 10^5$,对每个$k=1,...,m$,求有多少棵权值和为$k$的好树。答案对$998244353$取模。

    \vspace{1em}\pause
    【题解】设$g_i=[i\in c]$,$f_i$表示有多少颗权值之和为$k$的好树,有:
    \begin{equation*}
        f_m=\sum_{k=1}^M g_k \sum_{i=0}^{m-k} f_i f_{m-k-i}
    \end{equation*}

    为使递推成立,需要令$f_0=1$\pause ,设$F(x),G(x)$分别为$f,g$的普通型生成函数,有
    \begin{equation*}
        F(x)=G(x)F^2(x)+1
    \end{equation*}

    \pause
    因此
    \begin{equation*}
        F(x)=\frac{1\pm \sqrt{1-4G(x)}}{2G(x)}
    \end{equation*}

    应该选哪个解?
\end{frame}

\begin{frame}{例题选讲:小朋友和二叉树}
    \small
    \begin{equation*}
        F_1(x)=\frac{1+ \sqrt{1-4G(x)}}{2G(x)},\quad F_2(x)=\frac{1- \sqrt{1-4G(x)}}{2G(x)}
    \end{equation*}

    注意到$G(0)=0$,$F(0)=1$,根据上面的表达式,$x\to 0$时,$F_1(x)\to \infty$,因此舍弃$F_1$\pause ,从而
    \begin{equation*}
        F(x)=F_2(x)=\frac{2}{1+ \sqrt{1-4G(x)}}
    \end{equation*}

    多项式开根、求逆即可。
\end{frame}

\begin{frame}{例题选讲:简单连通图计数}
    \small
    求$n$个点带标号的无向简单连通图个数,$n\leq 10^5$,对$998244353$取模

    \vspace{1em}\pause
    【题解】设$f_n$为答案,$g_n$为$n$个点带标号简单图个数,显然$g_n=2^{\binom{n}{2}}$

    \pause
    枚举$1$号点所在的连通块大小,得到:
    \begin{equation*}
        g_n=\sum_{k=1}^n \binom{n-1}{k-1}f_k g_{n-k}
    \end{equation*}

    \pause
    展开组合数,得到
    \begin{equation*}
        \frac{g_n}{(n-1)!}=\sum_{k=1}^n \frac{f_k}{(k-1)!} \frac{g_{n-k}}{(n-k)!}
    \end{equation*}

    \pause
    设
    \begin{equation*}
        G(x)=\sum_{k=1}^\infty \frac{g_n x^n}{(n-1)!},\quad F(x)=\sum_{k=1}^\infty \frac{f_n x^n}{(k-1)!},\quad H(x)=\sum_{k=1}^\infty \frac{g_n x^n}{n!}
    \end{equation*}

    显然有$G(x)=F(x)H(x)$,$G(x)$与$H(x)$的各项系数直接求出,最后多项式求逆即可。
\end{frame}

\begin{frame}{例题选讲:竞赛图计数}
    \small
    求$n$个点强连通竞赛图的个数,对$998244353$取模,$n\leq 10^5$.

    \vspace{.5em}
    竞赛图:任意两点之间恰好有一条边的有向图。

    \vspace{1em}\pause
    【题解】设$f_n$表示$n$个点强连通竞赛图个数,$g_n$表示$n$个点竞赛图个数,显然$g_n=2^{\binom{n}{2}}$,另外我们认为$f_0=0,g_0=1$.

    \pause
    \vspace{.5em}
    考虑枚举拓扑序最小的强连通分量大小,注意,这个强连通分量与其他点的连边一定是从这个分量指出去的,所以竞赛图计数归结为两个子问题,从而有
    \begin{equation*}
        g_n=\sum_{i=1}^n \binom{n}{i} f_i g_{n-i} \implies \frac{g_n}{n!}=\sum_{i=0}^n \frac{f_i}{i!}\cdot \frac{g_{n-i}}{(n-i)!}
    \end{equation*}

    \pause
    \vspace{.5em}
    设$G(x),F(x)$分别是$g$与$f$的指数型生成函数,即得到$G(x)=F(x)G(x)+1$,于是
    \begin{equation*}
        F(x)=1-G^{-1}(x)
    \end{equation*}
\end{frame}

\begin{frame}{例题选讲:P5748 集合划分计数}
    \small
    一个$n$个元素的集合,将其分为任意多个子集,求方案数。

    $T$组数据,$T\leq 1000$,$n\leq 10^5$

    \vspace{1em}\pause
    【题解】 设$B_n$表示$n$个元素集合的划分方案数,考虑最后一个元素所在的集合,枚举该集合的大小、选取该集合中的元素,得到:
    \begin{equation*}
        B_{n+1}=\sum_{k=0}^{n}\binom{n}{k}B_{n-k}
    \end{equation*}

    \pause
    设$F(x)$是$B_n$的指数型生成函数,即:
    \begin{equation*}
        F(x)=\sum_{n= 0}^\infty B_n \frac{x^n}{n!}
    \end{equation*}
\end{frame}

\begin{frame}{例题选讲:P5748 集合划分计数}
    \small
    乘上$e^x$得到:
    \begin{align*}
        e^xF(x)&=\sum_{n= 0}^\infty \frac{x^n}{n!} \sum_{m= 0}^\infty B_m \frac{x^m}{m!}\\
        &= \sum_{n= 0}^\infty \sum_{m = 0}^n B_{n-m} \frac{x^n}{m!(n-m)!}\\
        &= \sum_{n= 0}^\infty x^n \sum_{m = 0}^n \binom{n}{m} B_{n-m} \frac{1}{n!}\\
        &= \sum_{n= 0}^\infty B_{n+1} \frac{x^n}{n!}= F'(x)
    \end{align*}

    \pause
    由于$F(0)=B_0=1$,可以得到$F(x)=e^{e^x-1}$,多项式exp求出$F(x)$各项系数即可。
\end{frame}

\begin{frame}{例题选讲:P4389 付公主的背包}
    \small
    有$n$种物品,体积为$v_i$,都有无限件。给定$m$,对于$s\in[1,m]$,求用这些物品恰好装满$s$体积的方案数。

    答案对$998244353$取模,$n,m\leq 10^5$.

    \vspace{1em}\pause
    【题解】设$f_{i,j}$表示用前$i$种物品恰好填满$j$体积的方案数,有dp:
    \begin{equation*}
        f_{i,j}=f_{i-1,j}+f_{i,j-v_i}
    \end{equation*}

    \pause
    设$F_i(x)$是$f_{i,1},f_{i,2},...$的普通型生成函数,由dp式得到:
    \begin{equation*}
        F_i(x)=F_{i-1}(x)+x^{v_i}F_i(x)\implies F_i(x)=F_{i-1}(x)\frac{1}{1-x^{v^i}}
    \end{equation*}
    
    于是可以得到$f_{n,1},f_{n,2},...$的普通型生成函数:
    \begin{equation*}
        F_n(x)=\prod_{i=1}^n \frac{1}{1-x^{v^i}}
    \end{equation*}
    
    这并不好算
\end{frame}

\begin{frame}{例题选讲:P4389 付公主的背包}
    \small
    考虑化乘法为加法,也就是取ln后再exp:
    \begin{equation*}
        F_n(x)=\exp \ln \prod_{i=1}^n \frac{1}{1-x^{v^i}}= \exp \sum_{i=1}^n \ln \frac{1}{1-x^{v^i}}
    \end{equation*}

    \pause
    对$\ln$项泰勒展开:
    \begin{equation*}
        F_n(x)=\exp \sum_{i=1}^n \sum_{j=1}^\infty \frac{x^{jv_i}}{j}
    \end{equation*}

    \pause
    事实上,我们只需在$(\text{mod}\; x^m)$下求系数,高次项可以忽略,因此
    \begin{equation*}
        F_n(x)=\exp \sum_{i=1}^n \sum_{j=1}^{\left\lfloor \frac{m}{v_i}\right\rfloor} \frac{x^{jv_i}}{j}
    \end{equation*}

    枚举的复杂度不能保证,还是不能做。
\end{frame}

\begin{frame}{例题选讲:P4389 付公主的背包}
    \small
    不妨将体积相同的物品视为一类,记$b_k$为体积为$k$的物品数量,于是合并同类项得到
    \begin{equation*}
        F_n(x)=\exp \sum_{k=1}^m b_k \sum_{j=1}^{\left\lfloor \frac{m}{k}\right\rfloor} \frac{x^{jk}}{j}
    \end{equation*}

    这个两重求和的复杂度由调和级数保证,是$O(m\log m)$的。最后再求$\exp$即可。
\end{frame}

\begin{frame}{例题选讲:节选自 P7289 Chasse Neige}
    \small
    给定$N$,对$n=1,2,...,N$,求有多少个长度为$n$的M型排列$\pi$,即:
    \begin{itemize}
        \item $\pi_1<\pi_2$
        \item $\pi_{n-1}>\pi_n$
        \item 每个数都交错,即:对于$i=2,3,...,n-1$,$\pi_i$要么比左右两个数都小,要么比左右两个数都大
    \end{itemize}

    答案对$998244353$取模,$N\leq 10^5$
\end{frame}

\begin{frame}{例题选讲:节选自 P7289 Chasse Neige}
    \small
    【题解】设$f_n$表示长度为$n$的M型排列个数\pause,枚举数字$n$放的位置$(i+1)$,选$i$个数放左边,剩下的放在右边,两边分别构成M型排列:
    \begin{equation*}
        f_n = \sum_{i \text{是奇数}} \binom{n-1}{i}f_i f_{n-1-i}
    \end{equation*}

    另外,显然$k$为偶数时$f_k=0$;特别地,我们需要$f_1=1$。\pause 设$F(x)$为$f$的指数型生成函数,有:
    \begin{equation*}
        F'(x)=F^2(x)+1
    \end{equation*}

    以及初始条件$F(0)=0$,下面来解这个方程。

    \pause
    \begin{equation*}
        \frac{\text{d} F}{\text{d}x}=F^2+1 \quad \implies \frac{\text{d} F}{F^2+1}=\text{d}x \quad \implies \int \frac{\text{d} F}{F^2+1}= \int \text{d}x
    \end{equation*}
    \pause
    \begin{equation*}
        \implies \arctan F = x+C \quad \implies F(x)=\tan(x+C)
    \end{equation*}

    由$F(0)=0$可以得到$C=0$,因此$F(x)=\tan x$,多项式求$\tan$即可。
\end{frame}


\section{常系数齐次线性递推}

\begin{frame}{问题引入}
    给定$f_1,...,f_k$,以及数列初值$a_1,...,a_k$,数列通项公式为
    \begin{equation*}
        a_n=\sum_{i=1}^k f_i a_{n-i}
    \end{equation*}

    求$a_n$,答案对$998244353$取模,$n\leq 10^9,k\leq 32000$
\end{frame}

\begin{frame}{矩阵形式}
    \small 
    我们把递推关系写成矩阵形式$\mathbf{y}_n=M\mathbf{y}_{n-1}$,展开为
    \begin{equation*}
        \begin{bmatrix}
            a_{n-k+1}\\
            a_{n-k+2}\\
            \vdots\\
            a_{n-1}\\
            a_{n}
        \end{bmatrix}=
        \begin{bmatrix}
            0 & 1 & 0 & \cdots & 0\\
            0 & 0 & 1 & \cdots & 0\\
            \vdots & \vdots & \vdots & \ddots & \vdots\\
            0 & 0 & 0 & \cdots & 1\\
            f_k & f_{k-1} & f_{k-2} & \cdots & f_1
        \end{bmatrix}
        \begin{bmatrix}
            a_{n-k}\\
            a_{n-k+1}\\
            \vdots\\
            a_{n-2}\\
            a_{n-1}
        \end{bmatrix}
    \end{equation*}

    $\mathbf{y}_k$是已知的,我们要求的就是$\mathbf{y}_{n+k-1}=M^{n-1}\mathbf{y}_k$的第一个分量。\pause $M$的特征多项式为
    \begin{equation*}
        p(x)=\det (xI-M) = x_k-\sum_{i=1}^k f_i x^{k-i}
    \end{equation*}

    \pause
    根据Cayley-Hamilton定理,有
    \begin{equation*}
        p(M)=O
    \end{equation*}
\end{frame}

\begin{frame}{进一步处理}
    \small 
    我们令$f(x)=x^{n-1}$,我们需要求的就是$f(M)=M^{n-1}$。\pause 借助多项式带余除法的思路,设
    \begin{equation*}
        f(x)=Q(x)p(x)+R(x)
    \end{equation*}
    
    其中$R(x)$的次数不超过$k-1$,也就是$R(x)=x^{n-1}\mod p(x)$,不妨记$R(x)=\sum_{i=0}^{k-1} r_i x^i$,\pause 将$M$代入,根据$p(M)=O$,得:
    \begin{equation*}
        M^n=f(M)=R(M)=\sum_{i=0}^{k-1} r_i M^i
    \end{equation*}

    怎么求$r_0,...,r_{k-1}$?

    \vspace{1em}
    \pause 快速幂取模。把快速幂里的运算全部换成多项式运算即可。当$k$较小时,多项式运算可以暴力实现,复杂度$O(k^2\log n)$;当$k$较大时需要NTT优化,复杂度$O(k\log k\log n)$
\end{frame}

\begin{frame}{进一步处理}
    \small 
    现在我们求出了$r_0,...,r_{k-1}$,所以
    \begin{equation*}
        \mathbf{y}_{n+k-1}=M^{n-1}\mathbf{y}_k=\sum_{i=0}^{k-1} r_i M^i\mathbf{y}_k=\sum_{i=0}^{k-1} r_i \mathbf{y}_{k+i}
    \end{equation*}

    \pause 我们只关心$\mathbf{y}_{n+k-1}$的第一个分量$a_n$,由上式子得
    \begin{equation*}
        a_n=\sum_{i=0}^{k-1} r_i a_{i+1}
    \end{equation*}

    $O(k)$即可求出。
\end{frame}

\begin{frame}{例题选讲:NOI2017 泳池}
    \small 
    有一个长度为$N$,高度为$1001$的网格图,每一个格子有$q$的概率是安全的,每个格子是否安全是独立事件。
    \vspace{.5em}

    如果一个子矩形中的所有格子都是安全的,且它的下边界贴着网格下边界,我们就说这个子矩形是安全的。
    \vspace{.5em}

    求最大安全子矩形面积为$K$的概率。
    \vspace{.5em}

    $n\leq 10^9$,$K\leq 10^3$,答案对$998244353$取模。
\end{frame}

\begin{frame}{例题选讲:NOI2017 泳池}
    \small 
    【题解】首先作前缀和差分:求出最大安全子矩形面积不超过$K$的概率,答案记作$Ans(K)$,再求出$Ans(K-1)$,相减即得到答案。
    \vspace{1em}

    \pause 设$f_n$表示底部宽度为$n$的区域内、右下角的格子是危险的、最大安全子矩形面积不超过$K$的概率;显然$Ans(K)=\frac{f_{n+1}}{1-q}$。
    \vspace{1em}

    \pause 再设$g_n$表示底部宽度为$n$的区域内、最底下一行格子全都是安全的、最大安全子矩形面积不超过$K$的概率。\pause 枚举最底下一行的最右边一个危险格子出现的位置,得到递推
    \begin{equation*}
        f_n=\sum_{i=0}^{\min(k-1,n)}(1-q)g_i f_{n-1-i}
    \end{equation*}

    \pause 如果$g_0,...,g_{k-1}$已知,这就是个常系数线性递推,数据范围是$k\leq 10^3$,所以直接用暴力方法$O(k^2\log n)$即可。下面考虑求$g$
\end{frame}

\begin{frame}{例题选讲:NOI2017 泳池}
    \small 
    设$h_{i,j}$表示宽为$i$的区域内、最底下$j$行全部安全、自底向上第$j+1$行存在危险格子、最大安全子矩形面积不超过$K$的概率\pause,显然
    \begin{equation*}
        g_i=\sum_{j=1}^{\left\lfloor\frac{k}{i}\right\rfloor} h_{i,j}
    \end{equation*}

    \pause 枚举自底向上第$j+1$行最左边的危险格子的位置,得到转移
    \begin{equation*}
        h_{i,j}=\sum_{s=1}^i q^j(1-q) \left(\sum_{t\geq j+1} h_{s-1,t}\right) \left(\sum_{t\geq j} h_{i-s,t}\right)
    \end{equation*}

    复杂度?\pause 注意到有效状态满足$ij\leq k$,所以调和级数保证了只有$O(k\log k)$个有效状态。再加一个后缀和优化就可以$O(k)$单次转移,总复杂度$O(k^2\log k)$
\end{frame}

\section{参考文献}

\begin{frame}[allowframebreaks]
    \bibliography{ref}
    \bibliographystyle{ieeetr}
    \nocite{*} % used here because no citation happens in slides
    % if there are too many try use:
    % \tiny\bibliographystyle{alpha}
\end{frame}


\begin{frame}
    \begin{center}
        {\Huge\calligra Thank You}
    \end{center}
\end{frame}

\end{document}
