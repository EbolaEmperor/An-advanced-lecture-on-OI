%%%%%%%%%%%%%%%%%%%%%%%%%%%%%%%%%%%%%%%%%%%%%%%%%%%%%%
% A Beamer template for Ritsumeikan University       %
% Author: Ming-Hao Xu (Xu Minghao)                   %
% Date:   April 2022.                                %
% LPPL Licensed.                                     %
%%%%%%%%%%%%%%%%%%%%%%%%%%%%%%%%%%%%%%%%%%%%%%%%%%%%%%

\documentclass{beamer}
\usepackage{hyperref}

\usepackage[UTF8]{ctex}
\usepackage[T1]{fontenc}

% other packages
\usepackage{latexsym,amsmath,xcolor,multicol,booktabs,calligra}
\usepackage{graphicx,pstricks,listings,stackengine}
\usefonttheme[onlymath]{serif}

% dummy text; remove it when working on this template
\usepackage{lipsum}

\author{Ebola}
\title{高级计数技巧:生成函数与多项式运算}
\institute{
    Institute of Mathematics, \\
    Zhejiang University.
}
\date{July, 2023}
\usepackage{Ritsumeikan}

% defs
\def\cmd#1{\texttt{\color{red}\footnotesize $\backslash$#1}}
\def\env#1{\texttt{\color{blue}\footnotesize #1}}
\definecolor{deepblue}{rgb}{0,0,0.5}
\definecolor{deepred}{rgb}{0.6,0,0}
\definecolor{deepgreen}{rgb}{0,0.5,0}
\definecolor{halfgray}{gray}{0.55}

\lstset{
    basicstyle=\ttfamily\tiny,
    keywordstyle=\bfseries\color{deepblue},
    emphstyle=\ttfamily\color{deepred},    % Custom highlighting style
    stringstyle=\color{deepgreen},
    numbers=left,
    numberstyle=\small\color{halfgray},
    rulesepcolor=\color{red!20!green!20!blue!20},
    frame=shadowbox,
}


\begin{document}

\begin{frame}
    \titlepage
\end{frame}

\begin{frame}
    \tableofcontents[sectionstyle=show,subsectionstyle=show/shaded/hide,subsubsectionstyle=show/shaded/hide]
\end{frame}

\section{生成函数}

\subsection{普通生成函数}

\begin{frame}{定义与例子}

对于一个数列$a_0,a_1,a_2,...$,它的普通生成函数为:
\begin{equation*}
    F(x)=\sum_n a_nx^n
\end{equation*}

\pause
\begin{itemize}
    \item $\{1,2,3\}$的普通生成函数为:$1+2x+3x^2$
    \item $\{1,1,1,...\}$的普通生成函数为:$\sum_{n=0}^\infty x^n$
\end{itemize}

\end{frame}

\begin{frame}{封闭形式}
    考虑$\{1,1,1,...\}$的普通生成函数:
    \begin{equation*}
        F(x)=\sum_{n=0}^\infty x^n
    \end{equation*}

    \pause
    注意到
    \begin{equation*}
        xF(x)+1=F(x)
    \end{equation*}

    因此
    \begin{equation*}
        F(x)=\frac{1}{1-x}
    \end{equation*}

    我们称这种没有求和符号的表达式为封闭形式
\end{frame}

\begin{frame}{封闭形式}
    求数列$\{1,0,1,0,...\}$的普通生成函数,并化为封闭形式

    \vspace{1em}\pause
    【解】
    \begin{equation*}
        F(x)=\sum_{n=0}^\infty x^{2n}=\frac{1}{1-x^2}
    \end{equation*}
\end{frame}

\begin{frame}{封闭形式}
    求数列$\{1,2,3,4,...\}$的普通生成函数,并化为封闭形式

    \vspace{1em}\pause
    【解】
    \begin{align*}
        F(x)&=\sum_{n=0}^\infty (n+1)x^{n}\\
        &= \sum_{n=0}^\infty (x^{n+1})'\\
        &= \left(\sum_{n=0}^\infty x^{n}\right)'\\
        &= \left(\frac{1}{1-x}\right)'\\
        &= \frac{1}{(1-x)^2}
    \end{align*}
\end{frame}

\begin{frame}{组合计数例子}
    \small
    假设你去买水果,一共要买$n$个,其中苹果只能买偶数个,西瓜最多买一个,梨可以买任意个。请问一共有多少种购买方案?

    \vspace{1em}\pause
    【解】设买$i$个苹果的方案数为$a_i$,事实上$a_i$仅在$i$为偶数时为$1$,否则为$0$,因此苹果的生成函数为:
    \begin{equation*}
        F(x)=1+x^2+x^4+...=\frac{1}{1-x^2}
    \end{equation*}

    \pause
    同理,西瓜和梨的生成函数为
    \begin{equation*}
        G(x)=1+x,\quad H(x)=1+x+x^2+...=\frac{1}{1-x}
    \end{equation*}

    \pause
    相乘得到
    \begin{equation*}
        F(x)G(x)H(x)=\frac{1}{(1-x)^2}=\sum_{n=0}^\infty (n+1)x^{n}
    \end{equation*}

    所以一共有$n+1$种购买方案。
\end{frame}

\subsection{指数生成函数}

\begin{frame}{定义与例子}

对于一个数列$a_0,a_1,a_2,...$,它的指数生成函数为:
\begin{equation*}
    F(x)=\sum_n a_n\frac{x^n}{n!}
\end{equation*}

\pause
\begin{itemize}
    \item $\{1,2,3\}$的指数生成函数为:$1+x+\frac{1}{2}x^2$
    \item $\{1,1,1,...\}$的指数生成函数为:$\sum_{n=0}^\infty \frac{x^n}{n!}$
\end{itemize}

\end{frame}

\begin{frame}{封闭形式}
    考虑$\{1,1,1,...\}$的指数生成函数:
    \begin{equation*}
        F(x)=\sum_{n=0}^\infty \frac{x^n}{n!}=e^x
    \end{equation*}
\end{frame}

\begin{frame}{封闭形式}
    求数列$\{1,0,1,0,...\}$的指数生成函数,并化为封闭形式

    \vspace{1em}\pause
    【解】
    \begin{align*}
        F(x)&=\sum_{n=0}^\infty \frac{x^{2n}}{(2n)!}\\
        &= \sum_{n=0}^\infty \frac{1+(-1)^n}{2} \frac{x^{n}}{n!}\\
        &= \frac{1}{2} \sum_{n=0}^\infty \frac{x^n}{n!} - \frac{1}{2} \sum_{n=0}^\infty \frac{(-x)^n}{n!}\\
        &= \frac{1}{2}\left(e^x+e^{-x}\right)
    \end{align*}
\end{frame}

\begin{frame}{例题选讲:森林计数}
    求$n$个点带标号、深度不超过$k$的森林一共有多少种。
\end{frame}

\section{快速数论变换(NTT)}

\begin{frame}{问题引入}
    给定多项式
    \begin{align*}
        f(x) &= a_0 + a_1x + a_2x^2 + \cdots + a_nx^n\\
        g(x) &= b_0 + b_1x + b_2x^2 + \cdots + b_mx^m
    \end{align*}
    求$h(x)=f(x)g(x)$的各项系数,对$998244353$取模。
\end{frame}

\begin{frame}{原根}
    注意:$998244353=119 \times 2^{23} + 1$

    \vspace{1em}
    我们说$3$是$998244353$的原根,因为$3^1,3^2,...,3^{998244352}$ 对$998244353$取模的结果两两不同。下面为了方便我们记$p=998244353$
    
    \vspace{1em}\pause
    根据费马小定理,$3^{p-1}\equiv 1\;(\text{mod}\;p)$,那么令$\omega_n=3^{119\times 2^{24-l}}$(其中$n=2^l$),我们会发现$\omega_{n}^{n}=1$,所以$\omega_n$可以作为$n$次单位根!

    \vspace{1em}\pause
    把FFT中的运算全部换成取模意义下的运算,再把$n$次单位复根替换成这里的$\omega_n$,就得到了NTT,它的性质就是取模意义下的FFT。
\end{frame}

\begin{frame}[fragile]{代码}
    \begin{lstlisting}{language=c++}
void NTT(int *a,bool INTT)
{
    for(int i=0;i<len;i++) r[i]=(r[i/2]/2)|((i&1)<<(l-1));
    for(int i=0;i<len;i++) if(i<r[i]) swap(a[i],a[r[i]]);
    for(int i=1;i<len;i<<=1)
    {
        int p=(i<<1);
        int wn=Pow(3,(Mod-1)/p);
        if(INTT) wn=Pow(wn,Mod-2);
        for(int j=0;j<len;j+=p)
        {
            int w=1;
            for(int k=0;k<i;k++)
            {
                int x=a[j+k],y=1ll*w*a[i+j+k]%Mod;
                a[j+k]=(x+y)%Mod;
                a[i+j+k]=(x-y+Mod)%Mod;
                w=1ll*w*wn%Mod;
            }
        }
    }
    //为了方便,我们通常把INTT的最后一步除以n也写进NTT函数里
    if(INTT) for(int i=0;i<len;i++) a[i]=1ll*a[i]*inv%Mod;
}
    \end{lstlisting}
\end{frame}

\section{多项式运算}

\begin{frame}{多项式牛顿迭代}

\small
给定多项式$g(x)$,求一个多项式$f(x)$,满足
\begin{equation}
    g(f(x))\equiv 0 \quad (\text{mod}\;x^n)
\end{equation}

注意:$(\text{mod}\;x^n)$的意思是只保留最低的$n$项。

\vspace{1em}\pause

考虑倍增。设$f_0(x)$是方程(1)在$(\text{mod}\;x^{\left\lceil\frac{n}{2}\right\rceil})$意义下的解,那么$f(x)-f_0(x)$的最低次项就是$x^{\left\lceil\frac{n}{2}\right\rceil}$项。考虑泰勒展开:
\begin{align*}
    g(f(x))&\equiv \sum_{i=0}^\infty \frac{g^{(i)}(f_0(x))}{i!}(f(x)-f_0(x))^i\quad (\text{mod}\;x^n)\\
    &\equiv g(f_0(x))+g'(f_0(x))(f(x)-f_0(x))\quad (\text{mod}\;x^n)
\end{align*}

\pause
由方程(1)得:
\begin{equation*}
    f(x)\equiv f_0(x) -\frac{g(f_0(x))}{g'(f_0(x))} \quad (\text{mod}\;x^n)
\end{equation*}

\end{frame}

\begin{frame}{多项式求逆}
    \small
    给定一个多项式$f(x)$,求一个多项式$g(x)$,使得
    \begin{equation}
        f(x)g(x) \equiv 1 \quad (\text{mod} x^n)
    \end{equation}

    \vspace{1em}\pause
    用多项式牛顿迭代的思路。设
    \begin{equation*}
        h(g(x))=\frac{1}{g(x)}-f(x)
    \end{equation*}

    \pause
    设$g_0(x)$是方程(2)在$(\text{mod}\;x^{\left\lceil\frac{n}{2}\right\rceil})$意义下的解,由牛顿迭代有:
    \begin{align*}
        g(x)&\equiv g_0(x) -\frac{h(g_0(x))}{h'(g_0(x))} \quad (\text{mod}\;x^n)\\
        &\equiv g_0(x) +\frac{\frac{1}{g_0(x)}-f(x)}{\frac{1}{g_0^2(x)}} \quad (\text{mod}\;x^n)\\
        &\equiv 2g_0(x)-g_0^2(x)f(x)\quad (\text{mod}\;x^n)
    \end{align*}

    FFT优化即可,复杂度$T(n)=T(n/2)+O(n\log n)=O(n\log n)$
\end{frame}

\begin{frame}{多项式开方}
    \small
    给定一个多项式$f(x)$,求一个多项式$g(x)$,使得
    \begin{equation}
        g^2(x) \equiv f(x) \quad (\text{mod} x^n)
    \end{equation}
    
    \vspace{1em}\pause
    用多项式牛顿迭代的思路。设
    \begin{equation*}
        h(g(x))=g^2(x)-f(x)
    \end{equation*}

    \pause
    设$g_0(x)$是方程(3)在$(\text{mod}\;x^{\left\lceil\frac{n}{2}\right\rceil})$意义下的解,由牛顿迭代有:
    \begin{align*}
        g(x)&\equiv g_0(x) -\frac{h(g_0(x))}{h'(g_0(x))} \quad (\text{mod}\;x^n)\\
        &\equiv g_0(x) -\frac{g_0^2(x)-f(x)}{2g_0(x)} \quad (\text{mod}\;x^n)\\
        &\equiv \frac{1}{2}g_0(x)+\frac{1}{2}g_0^{-1}(x)f(x)\quad (\text{mod}\;x^n)
    \end{align*}

    需要做一次多项式求逆和一次多项式乘法,复杂度$T(n)=T(n/2)+O(n\log n)=O(n\log n)$
\end{frame}

\begin{frame}{多项式除法(取模)}
    \small
    给定一个$n$次多项式$f(x)$和一个$m(\leq n)$次多项式$g(x)$,求多项式$Q(x),R(x)$,使得
    \begin{equation}
        f(x) \equiv Q(x)g(x)+R(x) \quad (\text{mod} x^n)
    \end{equation}

    且$\text{deg}\;R<m$(类似整数的带余除法)

    \vspace{1em}\pause
    考虑构造变换:
    \begin{equation*}
        f^R(x)=x^{n}f\left(\frac{1}{x}\right)
    \end{equation*}

    \pause
    将方程(4)中的$x$替换成$x^{-1}$,并且在两边乘上$x^n$,得到:
    \begin{equation*}
        x^nf\left(\frac{1}{x}\right)=x^{n-m}Q\left(\frac{1}{x}\right)x^m g\left(\frac{1}{x}\right) + x^{n-m+1}x^{m-1}R\left(\frac{1}{x}\right)
    \end{equation*}

    \pause
    \begin{equation*}
        \implies f^R(x) = Q^R(x)g^R(x)+x^{n-m+1}R^R(x)
    \end{equation*}
\end{frame}

\begin{frame}{多项式除法(取模)}
    \small
    \begin{equation*}
        f^R(x) = Q^R(x)g^R(x)+x^{n-m+1}R^R(x)
    \end{equation*}

    如果两边模掉$x^{n-m-1}$,就可以消除$R^R(x)$项的影响,而$Q^R(x)$的次数为$(n-m)<(n-m+1)$,所以$Q^R(x)$不受影响\pause,可见
    \begin{equation*}
        f^R(x)\equiv Q^R(x)g^R(x)\quad (\text{mod}\; x^{n-m+1})
    \end{equation*}

    多项式求逆得到$Q^R(x)$,反转得到$Q(x)$,回代方程(4)求出$R(x)$
\end{frame}

\begin{frame}{多项式ln}
    \small
    给定一个多项式$f(x)$,求一个多项式$g(x)$,使得
    \begin{equation}
        g(x) \equiv \ln f(x) \quad (\text{mod} x^n)
    \end{equation}
    
    \vspace{1em}\pause
    求导得
    \begin{equation*}
        g'(x) \equiv f'(x)f^{-1}(x) \quad (\text{mod} x^n)
    \end{equation*}

    依次进行求导、多项式求逆、多项式乘法、积分即可,复杂度$O(n\log n)$
\end{frame}

\begin{frame}{多项式exp}
    \small
    给定一个多项式$f(x)$,保证$[x^0]f(x)=0$,求一个多项式$g(x)$,使得
    \begin{equation}
        g(x) \equiv e^{f(x)}=\sum_{k=0}^\infty \frac{f^{k}(x)}{k!} \quad (\text{mod} x^n)
    \end{equation}
    
    \vspace{1em}\pause
    用多项式牛顿迭代的思路。设
    \begin{equation*}
        h(g(x))=\ln g(x)-f(x)
    \end{equation*}

    \pause
    设$g_0(x)$是方程(3)在$(\text{mod}\;x^{\left\lceil\frac{n}{2}\right\rceil})$意义下的解,由牛顿迭代有:
    \begin{align*}
        g(x)&\equiv g_0(x) -\frac{h(g_0(x))}{h'(g_0(x))} \quad (\text{mod}\;x^n)\\
        &\equiv g_0(x) -\frac{\ln g_0(x)-f(x)}{g_0^{-1}(x)} \quad (\text{mod}\;x^n)\\
        &\equiv g_0(x) \left(1-\ln g_0(x) + f(x)\right)\quad (\text{mod}\;x^n)
    \end{align*}

    需要做一次多项式ln和一次多项式乘法,复杂度$T(n)=T(n/2)+O(n\log n)=O(n\log n)$
\end{frame}

\begin{frame}{多项式的幂}
    \small
    给定正整数$k$和一个多项式$f(x)$,保证$[x^0]f(x)=1$,求一个多项式$g(x)$,使得
    \begin{equation}
        g(x) \equiv \left(f(x)\right)^k \quad (\text{mod} x^n)
    \end{equation}
    
    \vspace{1em}\pause
    注意到
    \begin{equation*}
        g(x)=e^{k\ln f(x)}
    \end{equation*}
\end{frame}

\begin{frame}{例题选讲}
    \small
    求$n$个点带标号的无向简单连通图个数,$n\leq 10^5$,对$998244353$取模

    \vspace{1em}\pause
    【题解】设$f_n$为答案,$g_n$为$n$个点带标号简单图个数,显然$g_n=2^{\binom{n}{2}}$

    \pause
    枚举$1$号点所在的连通块大小,得到:
    \begin{equation*}
        g_n=\sum_{k=1}^n \binom{n-1}{k-1}f_k g_{n-k}
    \end{equation*}

    \pause
    展开组合数,得到
    \begin{equation*}
        \frac{g_n}{(n-1)!}=\sum_{k=1}^n \frac{f_k}{(k-1)!} \frac{g_{n-k}}{(n-k)!}
    \end{equation*}

    \pause
    设
    \begin{equation*}
        G(x)=\sum_{k=1}^\infty \frac{g_n x^n}{(n-1)!},\quad F(x)=\sum_{k=1}^\infty \frac{f_n x^n}{(k-1)!},\quad H(x)=\sum_{k=1}^\infty \frac{g_n x^n}{n!}
    \end{equation*}

    显然有$G(x)=F(x)H(x)$,$G(x)$与$H(x)$的各项系数直接求出,最后多项式求逆即可。
\end{frame}

\begin{frame}{例题选讲:P5748 集合划分计数}
    \small
    一个$n$个元素的集合,将其分为任意多个子集,求方案数。

    $T$组数据,$T\leq 1000$,$n\leq 10^5$

    \vspace{1em}\pause
    【题解】 设$B_n$表示$n$个元素集合的划分方案数,考虑最后一个元素所在的集合,枚举该集合的大小、选取该集合中的元素,得到:
    \begin{equation*}
        B_{n+1}=\sum_{k=0}^{n}\binom{n}{k}B_{n-k}
    \end{equation*}

    \pause
    设$F(x)$是$B_n$的指数型生成函数,即:
    \begin{equation*}
        F(x)=\sum_{n= 0}^\infty B_n \frac{x^n}{n!}
    \end{equation*}
\end{frame}

\begin{frame}{例题选讲:P5748 集合划分计数}
    \small
    乘上$e^x$得到:
    \begin{align*}
        e^xF(x)&=\sum_{n= 0}^\infty \frac{x^n}{n!} \sum_{m= 0}^\infty B_m \frac{x^m}{m!}\\
        &= \sum_{n= 0}^\infty \sum_{m = 0}^n B_{n-m} \frac{x^n}{m!(n-m)!}\\
        &= \sum_{n= 0}^\infty x^n \sum_{m = 0}^n \binom{n}{m} B_{n-m} \frac{1}{n!}\\
        &= \sum_{n= 0}^\infty B_{n+1} \frac{x^n}{n!}= F'(x)
    \end{align*}

    \pause
    由于$F(0)=B_0=1$,可以得到$F(x)=e^{e^x-1}$,多项式exp求出$F(x)$各项系数即可。
\end{frame}

\section{多项式的其它应用}

\subsection{多点求值与快速插值}

\subsection{常系数齐次线性递推}

\section{参考文献}

\begin{frame}[allowframebreaks]
    \bibliography{ref}
    \bibliographystyle{ieeetr}
    \nocite{*} % used here because no citation happens in slides
    % if there are too many try use:
    % \tiny\bibliographystyle{alpha}
\end{frame}


\begin{frame}
    \begin{center}
        {\Huge\calligra Thank You}
    \end{center}
\end{frame}

\end{document}